\begin{enumerate}
  \item Выполнить минимизацию булевых функций, представить функции различных базисах – основном логическом базисе (О), в базисе Шеффера (Ш), Пирса (П) или Жегалкина(Ж) в соответствии с вариантом, после чего построить схему в системе Logisim и выполнить проверку.
  \item Построить четырехразрядный полный сумматор, складывающий 2 двоичных четырехразрядных числа и учитывающий единицу переноса. Построить схему сумматора в Logisim, проверить его работоспособность.
  \item Построить схемы прямого (на +1) и обратного (на -1) 4-разрядных двоичных счетчиков на счетных (T) триггерах. Построить схемы счетчиков в Logisim, проверить их работоспособность.
  \item Гирлянда. На базе счетчика, дешифратора построить схему, включающие светодиоды в определенном порядке в зависимости от варианта. Построить схему в Logisim, проверить его работоспособность.
  \item Построить схему дешифратора семисегментного индикатора.
  \item Построить схему 4-разрядного последовательного сдвигового регистра.  Сдвиг в любую сторону, запись последовательная по битам, чтение параллельное.
  \item Построить схему последовательного (shift-add) 8-разрядного умножителя на сдвиговом регистре.
  \item Построить схему 64-разрядного сумматора с ускоренным переносом.
\end{enumerate}
