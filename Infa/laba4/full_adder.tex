\documentclass [15pt,a4paper,twoside]{article}
\usepackage[russian,shorthands=off]{babel}        % shorhands=off is required for babel french in combination with tikz karnaugh....
\usepackage[utf8x]{inputenc}
\usepackage[T1]{fontenc}
\usepackage{amsmath}
\usepackage{geometry}
\geometry{verbose,a4paper, tmargin=3.5cm,bmargin=3.5cm,lmargin=2.5cm,rmargin=2.5cm,headsep=1cm,footskip=1.5cm}
\usepackage{fancyhdr}
\usepackage{colortbl}
\usepackage[dvipsnames]{xcolor}
\usepackage{tikz -timing}
\usepackage{tikz}
\usetikzlibrary{karnaugh}
\pagestyle{fancy}

\definecolor{LogisimKMapColor0}{RGB}{128,0,0}
\definecolor{LogisimKMapColor1}{RGB}{230,25,75}
\definecolor{LogisimKMapColor2}{RGB}{250,190,190}
\definecolor{LogisimKMapColor3}{RGB}{170,110,40}
\definecolor{LogisimKMapColor4}{RGB}{245,130,48}
\definecolor{LogisimKMapColor5}{RGB}{255,215,180}
\definecolor{LogisimKMapColor6}{RGB}{128,128,0}
\definecolor{LogisimKMapColor7}{RGB}{255,255,25}
\definecolor{LogisimKMapColor8}{RGB}{210,245,60}
\definecolor{LogisimKMapColor9}{RGB}{0,0,128}
\definecolor{LogisimKMapColor10}{RGB}{145,30,180}
\definecolor{LogisimKMapColor11}{RGB}{60,180,175}
\definecolor{LogisimKMapColor12}{RGB}{0,130,203}
\definecolor{LogisimKMapColor13}{RGB}{230,190,255}
\definecolor{LogisimKMapColor14}{RGB}{170,255,195}
\definecolor{LogisimKMapColor15}{RGB}{240,50,230}

\fancyhead{}
\fancyhead[C] {Сгенерированный документ Logisim-evolution на Mon Dec 15 16:06:11 MSK 2025}
\fancyfoot[C] {\thepage}
\renewcommand{\headrulewidth}{0.4pt}
\renewcommand{\footrulewidth}{0.4pt}

\makeatother

\begin{document}
\section{Введение}
Этот документ был разработан в результате логистической эволюции. Любая часть исходных текстов TeX может быть использована в ваших собственных документах без каких-либо проблем. Если вы хотите использовать все/части сгенерированных TeX-источников, не забудьте (1) включить необходимые пакеты и (2) отметить, что этот источник был создан в результате логистической эволюции.
%===============================================================================
\section{Таблица правды}
Таблица может быть на пути к большому размеру для отображения на странице. Во время генерации не было выполнено вычислений по размеру таблицы относительно ширины/высоты страницы.
%-------------------------------------------------------------------------------
\subsection{Сжатая таблица истинности}
\begin{center}
\begin{tabular}{ccc|cc}
$A$&$B$&$C_in$&$S$&$C_out$\\
\hline
$0$&$0$&$0$&$0$&$0$\\
$0$&$0$&$1$&$1$&$0$\\
$0$&$1$&$0$&$1$&$0$\\
$0$&$1$&$1$&$0$&$1$\\
$1$&$0$&$0$&$1$&$0$\\
$1$&$0$&$1$&$0$&$1$\\
$1$&$1$&$0$&$0$&$1$\\
$1$&$1$&$1$&$1$&$1$\\

\end{tabular}
\end{center}
%-------------------------------------------------------------------------------
\subsection{Полная таблица истинности}
\begin{center}
\begin{tabular}{ccc|cc}
$A$&$B$&$C_in$&$S$&$C_out$\\
\hline
$0$&$0$&$0$&$0$&$0$\\
$0$&$0$&$1$&$1$&$0$\\
$0$&$1$&$0$&$1$&$0$\\
$0$&$1$&$1$&$0$&$1$\\
$1$&$0$&$0$&$1$&$0$\\
$1$&$0$&$1$&$0$&$1$\\
$1$&$1$&$0$&$0$&$1$\\
$1$&$1$&$1$&$1$&$1$\\

\end{tabular}
\end{center}
%===============================================================================
\section{Диаграммы Карнауга}
В данном разделе представлены различные варианты диаграмм Карнауга данных функций.
%-------------------------------------------------------------------------------
\subsection{Пустые диаграммы Карнауга}
\begin{center}
\begin{tikzpicture}[karnaugh,disable bars,x=1\kmunitlength,y=1\kmunitlength,kmbar left sep=1\kmunitlength,grp/.style n args={4}{#1,fill=#1!30,minimum width= #2\kmunitlength,minimum height=#3\kmunitlength,rounded corners=0.2\kmunitlength,fill opacity=0.6,rectangle,draw}]
\karnaughmap{3}{$S$}{{$B$}{$A$}{$C_in$}}{}{
\draw[kmbox] (-0.5,2.5)
   node[below left]{$A$}
   node[above right]{$B$, $C_in$} +(-0.2,0.2)
   node[above left]{$S$};\draw (0,2) -- (-0.7,2.7);
\foreach \x/\1 in %
{0/00,1/01,2/11,3/10} {
   \node at (\x+0.5,2.2) {\1};
}
\foreach \y/\1 in %
{0/0,1/1} {
   \node at (-0.4,-0.5-\y+2) {\1};
}
}
\end{tikzpicture}
\end{center}
\begin{center}
\begin{tikzpicture}[karnaugh,disable bars,x=1\kmunitlength,y=1\kmunitlength,kmbar left sep=1\kmunitlength,grp/.style n args={4}{#1,fill=#1!30,minimum width= #2\kmunitlength,minimum height=#3\kmunitlength,rounded corners=0.2\kmunitlength,fill opacity=0.6,rectangle,draw}]
\karnaughmap{3}{$C_out$}{{$B$}{$A$}{$C_in$}}{}{
\draw[kmbox] (-0.5,2.5)
   node[below left]{$A$}
   node[above right]{$B$, $C_in$} +(-0.2,0.2)
   node[above left]{$C_out$};\draw (0,2) -- (-0.7,2.7);
\foreach \x/\1 in %
{0/00,1/01,2/11,3/10} {
   \node at (\x+0.5,2.2) {\1};
}
\foreach \y/\1 in %
{0/0,1/1} {
   \node at (-0.4,-0.5-\y+2) {\1};
}
}
\end{tikzpicture}
\end{center}
%-------------------------------------------------------------------------------
\subsection{Заполнил диаграммы Карнауга.}
\begin{center}
\begin{tikzpicture}[karnaugh,disable bars,x=1\kmunitlength,y=1\kmunitlength,kmbar left sep=1\kmunitlength,grp/.style n args={4}{#1,fill=#1!30,minimum width= #2\kmunitlength,minimum height=#3\kmunitlength,rounded corners=0.2\kmunitlength,fill opacity=0.6,rectangle,draw}]
\karnaughmap{3}{$S$}{{$B$}{$A$}{$C_in$}}
{01101001}{
\draw[kmbox] (-0.5,2.5)
   node[below left]{$A$}
   node[above right]{$B$, $C_in$} +(-0.2,0.2)
   node[above left]{$S$};\draw (0,2) -- (-0.7,2.7);
\foreach \x/\1 in %
{0/00,1/01,2/11,3/10} {
   \node at (\x+0.5,2.2) {\1};
}
\foreach \y/\1 in %
{0/0,1/1} {
   \node at (-0.4,-0.5-\y+2) {\1};
}
}
\end{tikzpicture}
\end{center}
\begin{center}
\begin{tikzpicture}[karnaugh,disable bars,x=1\kmunitlength,y=1\kmunitlength,kmbar left sep=1\kmunitlength,grp/.style n args={4}{#1,fill=#1!30,minimum width= #2\kmunitlength,minimum height=#3\kmunitlength,rounded corners=0.2\kmunitlength,fill opacity=0.6,rectangle,draw}]
\karnaughmap{3}{$C_out$}{{$B$}{$A$}{$C_in$}}
{00010111}{
\draw[kmbox] (-0.5,2.5)
   node[below left]{$A$}
   node[above right]{$B$, $C_in$} +(-0.2,0.2)
   node[above left]{$C_out$};\draw (0,2) -- (-0.7,2.7);
\foreach \x/\1 in %
{0/00,1/01,2/11,3/10} {
   \node at (\x+0.5,2.2) {\1};
}
\foreach \y/\1 in %
{0/0,1/1} {
   \node at (-0.4,-0.5-\y+2) {\1};
}
}
\end{tikzpicture}
\end{center}
%-------------------------------------------------------------------------------
\subsection{Заполненные диаграммы Карнауга с крышками}
\begin{center}
\begin{tikzpicture}[karnaugh,disable bars,x=1\kmunitlength,y=1\kmunitlength,kmbar left sep=1\kmunitlength,grp/.style n args={4}{#1,fill=#1!30,minimum width= #2\kmunitlength,minimum height=#3\kmunitlength,rounded corners=0.2\kmunitlength,fill opacity=0.6,rectangle,draw}]
\karnaughmap{3}{$S$}{{$B$}{$A$}{$C_in$}}
{01101001}{
\draw[kmbox] (-0.5,2.5)
   node[below left]{$A$}
   node[above right]{$B$, $C_in$} +(-0.2,0.2)
   node[above left]{$S$};\draw (0,2) -- (-0.7,2.7);
\foreach \x/\1 in %
{0/00,1/01,2/11,3/10} {
   \node at (\x+0.5,2.2) {\1};
}
\foreach \y/\1 in %
{0/0,1/1} {
   \node at (-0.4,-0.5-\y+2) {\1};
}
   \node[grp={LogisimKMapColor0}{0.8}{0.8}](n0) at(1.5,1.5) {};
   \node[grp={LogisimKMapColor1}{0.8}{0.8}](n1) at(3.5,1.5) {};
   \node[grp={LogisimKMapColor2}{0.8}{0.8}](n2) at(0.5,0.5) {};
   \node[grp={LogisimKMapColor3}{0.8}{0.8}](n3) at(2.5,0.5) {};
}
\end{tikzpicture}
\end{center}
\begin{center}
\begin{tikzpicture}[karnaugh,disable bars,x=1\kmunitlength,y=1\kmunitlength,kmbar left sep=1\kmunitlength,grp/.style n args={4}{#1,fill=#1!30,minimum width= #2\kmunitlength,minimum height=#3\kmunitlength,rounded corners=0.2\kmunitlength,fill opacity=0.6,rectangle,draw}]
\karnaughmap{3}{$C_out$}{{$B$}{$A$}{$C_in$}}
{00010111}{
\draw[kmbox] (-0.5,2.5)
   node[below left]{$A$}
   node[above right]{$B$, $C_in$} +(-0.2,0.2)
   node[above left]{$C_out$};\draw (0,2) -- (-0.7,2.7);
\foreach \x/\1 in %
{0/00,1/01,2/11,3/10} {
   \node at (\x+0.5,2.2) {\1};
}
\foreach \y/\1 in %
{0/0,1/1} {
   \node at (-0.4,-0.5-\y+2) {\1};
}
   \node[grp={LogisimKMapColor0}{0.8}{1.8}](n0) at(2.5,1) {};
   \node[grp={LogisimKMapColor1}{1.8}{0.8}](n1) at(2,0.5) {};
   \node[grp={LogisimKMapColor2}{1.8}{0.8}](n2) at(3,0.5) {};
}
\end{tikzpicture}
\end{center}
%===============================================================================
\section{Минимальные выражения}
$S =  \overline{A}  \cdot  \overline{B}  \cdot C_in+ \overline{A}  \cdot B \cdot  \overline{C_in} +A \cdot  \overline{B}  \cdot  \overline{C_in} +A \cdot B \cdot C_in$~\\
$C_out = B \cdot C_in+A \cdot C_in+A \cdot B$~\\
\end{document}
