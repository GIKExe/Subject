\documentclass[oneside,a4paper,14pt]{extarticle} %размер шрифта 14
\usepackage[T1,T2A]{fontenc}
\usepackage[
    a4paper,
    letterpaper,
    top=2cm,
    bottom=2cm,
    left=2.5cm,
    right=1.5cm
]{geometry} 
\usepackage[utf8]{inputenc} %кодировка текста
\usepackage[russian]{babel} %поддержка русского языка
\usepackage{textcomp} %текстовые символы
\usepackage{indentfirst} %корректировка отступов
\usepackage{graphicx} %работа с изображениям
\usepackage{mwe} % for blindtext and example-image-a in example
\usepackage{wrapfig}
\usepackage{caption}
\usepackage{amsmath}  % для формул и символов
\usepackage{amsfonts}
\usepackage{amsthm}
\usepackage{graphicx}
\usepackage[all]{xy}
\usepackage[breaklinks]{hyperref}
%%размеркегляузаголоковразделов 
\usepackage{titlesec} 
\titleformat{\section}
{\normalsize\bfseries} 
{\thesection} {1em}{} 
\titleformat{\subsection}
{\normalsize\bfseries}
{\thesubsection} {1em}{} 
\titleformat{\subsubsection}
{\normalsize\bfseries}
{\thesubsection} {1em}{}

\renewcommand\baselinestretch{1.33}\normalsize % межстрочный интервал
\setlength{\parindent}{1.25cm}
\usepackage{indentfirst}


\begin{document}
    \newpage\thispagestyle{empty}
    \begin{center}
        МИНИСТЕРСТВО НАУКИ И ВЫСШЕГО ОБРАЗОВАНИЯ\\
            РОССИЙСКОЙ ФЕДЕРАЦИИ
            ФЕДЕРАЛЬНОЕ ГОСУДАРСТВЕННОЕ БЮДЖЕТНОЕ\\
            ОБРАЗОВАТЕЛЬНОЕ
            УЧРЕЖДЕНИЕ ВЫСШЕГО ОБРАЗОВАНИЯ\\
            «ВЯТСКИЙ ГОСУДАРСТВЕННЫЙ УНИВЕРСИТЕТ»\\
            Институт математики и информационных систем\\
            Факультет автоматики и вычислительной техники\\
            Кафедра электронных вычислительных машин
    \end{center}
    \vspace{20mm}
    
    \begin{center}
        Отчёт по лабораторной работе №1\\
        по дисциплине\\
        <<Информатика>>\\
        <<Арифметические операции в системах счисления>>\\
    \end{center}
    \vspace{48mm}
    
    Выполнил студент гр. ИВТб-1303-06-00 \hspace{10mm} \rule[-0,5mm]{23mm}{0.15mm}\,/Гортоломей И.К./
    
    
    Проверил доцент кафедры ЭВМ \hfill  \rule[-0,5mm]{30mm}{0.15mm}\,/Коржавина А.С./
    
    \vfill
    \begin{center}
        Киров\\
        2025
    \end{center}

    \newpage\thispagestyle{empty}
    
   \section{Цель}
   
Цель лабораторной работы: закрепить на практике знания о кодировании числовой информации, представления чисел в позиционных и непозиционных системах счисления, о выполнении арифметических операций сложения и умножения чисел в позиционных и непозиционных системах счисления.
   
    \section*{Задание}
    
 \begin{enumerate}

\item В каждом варианте даны X и K.  Выполнить перевод числа X из десятичной системы счисления в систему счисления с основанием K в соответствии с вариантом. 
\\ Проверить полученные результаты.

\item Выполнить перевод числа из десятичной системы счисления в восьмеричную (8СС) систему счисления, из 8СС в  двоичную систему счисления (2СС), из двоичной -- шестнадцатеричную (16СС), из 16СС обратно в десятичную в соответствии с вариантом (10СС->8CC->2CC->16CC->10CC)
\\ Проверить полученные результаты.

\item В каждом варианте дана пара чисел (X и Y).  Выполнить перевод чисел из десятичной системы счисления в двоичную систему счисления (2СС), выполнить сложение и умножение чисел.
\\ Проверить полученные результаты.

\item В каждом варианте даны 2 пары чисел (X3 и Y3, X4 и Y4). Выполнить перевод чисел из десятичной системы счисления в 16СС.  Выполнить сложение шестнадцатеричных чисел в соответствии с вариантом.
\\ Проверить полученные результаты.

\item Выполнить перевод в двоично-десятичную систему счисления в соответствии с вариантом. В каждом варианте даны 3 числа. Представить первое число в коде 8421 (код с естественными весами), второе число в коде 8421+3, третье число в коде 2421.
\\ Проверить полученные результаты.

\item Выполнить перевод в систему остаточных классов в соответствии с вариантом. В каждом варианте даны 2 числа (А и В) и соответствующие им базисы. Выполнить сложение и умножение чисел.
\\ Проверить полученные результаты.

\item Выполнить перевод в троичную симметричную систему счисления в соответствии с вариантом. В каждом варианте даны 2 числа. Выполнить сложение чисел.
\\ Проверить полученные результаты.

\item Выполнить перевод в двоично-десятичную систему счисления в соответствии с вариантом. В каждом варианте даны 2 пары чисел. Представить первую пару чисел в коде 8421 (код с естественными весами), вторую пару в коде 8421+3. Выполнить сложение чисел.
\\ Проверить полученные результаты.

 \end{enumerate}
    \section*{Решение}
    
    \begin{enumerate}
    
    \item перевод $X=101$ в 8-ичную СС
    \[
        \begin{array}[c]{ll}
            101=12\cdot 8 + 5, &\Rightarrow b_{0}=5, \\
            12=1\cdot 8 + 4,   &\Rightarrow b_{1}=4, \\
            1=0\cdot 8 + 1,    &\Rightarrow b_{2}=1, \\
        \end{array}
    \]
    \\ $X\equiv 145_{8}$

    \item перевод $X=152.49$ в 8-ичную СС
    \\ Целая часть
    \[
        \begin{array}[c]{ll}
            152=19\cdot 8 + 0, &\Rightarrow b_{0}=0, \\
            19=2\cdot 8 + 3,   &\Rightarrow b_{1}=3, \\
            2=0\cdot 8 + 2,    &\Rightarrow b_{2}=2, \\
        \end{array}
    \]
    \\ Дробная часть
    \[
        \begin{array}[c]{ll}
            0.49\cdot 8=3.92,   &\Rightarrow b_{-1}=3, \\
            0.92\cdot 8=7.36,   &\Rightarrow b_{-2}=7, \\
            0.36\cdot 8=2.88,   &\Rightarrow b_{-3}=2, \\
            \cdots
        \end{array}
    \]
    \\ $X\equiv 230.372_{8}$  
    \\ из 8-ичной в 2-ичную СС будем переводить триадами
    \\ $230.372_{8}\equiv 010\ 011\ 000\ .\ 011\ 111\ 010=10011000.01111101$
    \\ из 2-ичной в 16-ичную СС будем переводить тетрадами
    \\ $1001\ 1000\ .\ 0111\ 1101_{2}\equiv 98.7D$
    \\ из 16-ичной в 10-ичную СС будем переводить умножением
    \\ $9\cdot16^1 + 8\cdot16^0 + 7\cdot16^{-1} + D\cdot16^{-2}=152.488$

    \item перевод $X=8.1$ и $Y=8$ в 2-ичную СС
    \\ Дробная часть X
    \[
        \begin{array}[c]{ll}
            0.1\cdot 2=0.2,   &\Rightarrow b_{-1}=0, \\
            0.2\cdot 2=0.4,   &\Rightarrow b_{-2}=0, \\
            0.4\cdot 2=0.8,   &\Rightarrow b_{-3}=0, \\
            0.8\cdot 2=1.6,   &\Rightarrow b_{-4}=1, \\
            0.6\cdot 2=1.2,   &\Rightarrow b_{-5}=1, \\
            \cdots
        \end{array}
    \]
    \\ Целая часть X и Y 
    \\ $8=2^3=1000_2$
    \[
        {\entrymodifiers={}
            {\xymatrix@=1pc{
                X& \equiv
                    & &1&0&0&0&.&0&0&0&1&1\\
                Y& \equiv 
                    & &1&0&0&0&.&0&0&0&0&0\\
                \xleftarrow{c}
                 &  &_1& & & & & & & & & & \\
                X+Y&\equiv
                    &1
                      &0\ar[ul]
                      &0
                      &0
                      &0
                      &.
                      &0
                      &0
                      &0
                      &1 &1
            }}
        }
    \]
    \[    
        \begin{tabular}{{c}{c}}
        \texttt{$\times$ }&
        \begin{tabular}{c}
           \texttt{~~~1000.0001100000}\\
           \texttt{~~~1000.0000000000}\\
        \end{tabular} \\ 
        \hline
         & \texttt{~~~~~~~.0000000000}\\
         & \texttt{~~~~~~0.000000000~}\\
         & \texttt{~~~~~00.00000000~~}\\
         & \texttt{~~~~000.0000000~~~}\\
         & \texttt{~~~0000.000000~~~~}\\
         & \texttt{~~00000.00000~~~~~}\\
         & \texttt{~000000.0000~~~~~~}\\
         & \texttt{1000000.110~~~~~~~}\\
        \hline
         & \texttt{1000000.11~~~~~~~~}
        \end{tabular}
    \] 
    $X+Y\equiv10000.00011_2$, $X\cdot Y\equiv 1000 000.11_2$ \\
    $10000.00011_2=2^4+2^{-4}+2^{-5}=16.09375<16.1$ \\
    $1000 000.11_2=2^6+2^{-1}+2^{-2}=64.75<64.8$

    
    \item $X3=87$, $Y3=70$ \\
    \[\begin{array}[c]{ll}
        87=5\cdot 16 + 7, &\Rightarrow b_{0}=5, \\
        7=0\cdot 16 + 7,  &\Rightarrow b_{1}=7, \\
    \end{array}\]
    
    $X3=57_{16}$
    \[\begin{array}[c]{ll}
        70=4\cdot 16 + 6, &\Rightarrow b_{0}=4, \\
        6=0\cdot 16 + 6,  &\Rightarrow b_{1}=6, \\
    \end{array}\]
    $Y3=46_{16}$
    \[
        {\entrymodifiers={}
            {\xymatrix@=1pc{
                X3& =
                    &5&7\\
                Y3& = 
                    &4&6\\
                \xleftarrow{c}
                 &  & & & \\
                X3+Y3& =
                    &9 &D
            }}
        }
    \] \newpage $X4=69$, $Y4=125$
    \[\begin{array}[c]{ll}
        69=4\cdot 16 + 5, &\Rightarrow b_{0}=4, \\
        5=0\cdot 16 + 5,  &\Rightarrow b_{1}=5, \\
    \end{array}\]
    
    $X4=45_{16}$
    \[\begin{array}[c]{ll}
        125=7\cdot 16 + 13, &\Rightarrow b_{0}=7, \\
        13=0\cdot 16 + 13,  &\Rightarrow b_{1}=13, \\
    \end{array}\]
    $Y4=7D_{16}$
    \[
        {\entrymodifiers={}
            {\xymatrix@=1pc{
                X4& =
                    &4&5\\
                Y4& = 
                    &7&D\\
                \xleftarrow{c}
                 &  &_1 & & \\
                X4+Y4& =
                    &C &2\ar[ul]
            }}
        }
    \]

    \item в кодировке "8421"
    \\ $72_{10}=01110010_{2-10}$
    \\ в кодировке "8421+3"
    \\ $72_{10}=1010 0101_{2-10}$
    \\ в кодировке "2421"
    \\ $62_{10}=1100 0010_{2-10}$

    \item перевод \{9,11,13,15\}, $A=114$, $B=111$ в СОК
    \\ \% - остаток от деления
    \\ $A=(114\%9, 114\%11, 114\%13, 114\%15)=(6,4,10,9)$
    \\ $B=(111\%9, 111\%11, 111\%13, 111\%15)=(3,1,7,6)$
    \\ $A+B=(9\%9, 5\%11, 17\%13, 15\%15)=(0,5,4,0)$
    \\ $A\cdot B=(18\%9, 4\%11, 70\%13, 54\%15)=(0,4,5,9)$

    \item перевод $A=75$ в 3-ичную Симметричную СС
    \[
        \begin{array}[c]{lll}
            75=25\cdot 3 + 0, &\Rightarrow r_{0}=0,& b_{0}=0, \\
            25=8\cdot 3 + 1, &\Rightarrow r_{1}=1,& b_{1}=p, \\
            8=3\cdot 3 - 1, &\Rightarrow r_{2}=-1,& b_{2}=n, \\
            3=1\cdot 3 + 0, &\Rightarrow r_{3}=0,& b_{3}=0, \\
            1=0\cdot 3 + 1, &\Rightarrow r_{4}=1,& b_{4}=p, \\
        \end{array}
    \]
    \\ перевод $B=-125$ в 3-ичную Симметричную СС
    \[
        \begin{array}[c]{lll}
            125=42\cdot 3 - 1, &\Rightarrow r_{0}=-1,& b_{0}=n, \\
            42=14\cdot 3 + 0, &\Rightarrow r_{1}=0,& b_{1}=0, \\
            14=5\cdot 3 - 1, &\Rightarrow r_{2}=-1,& b_{2}=n, \\
            5=2\cdot 3 - 1, &\Rightarrow r_{3}=-1,& b_{3}=n, \\
            2=1\cdot 3 - 1, &\Rightarrow r_{4}=-1,& b_{4}=n, \\
            1=0\cdot 3 + 1, &\Rightarrow r_{5}=1,& b_{5}=p, \\
        \end{array}
    \]
     \\ сложение A и B
     \[
        {\entrymodifiers={}
            {\xymatrix@=1pc{
                X& \equiv & &0 &p &0 &n &p &0 \\
                B& \equiv & &n &p &p &p &0 &p \\
                \xleftarrow{c}
                 &  & &_p & & & & & \\
                X+Y&\equiv
                   & &0 &n\ar[ul] &p &0 &p &p
            }}
        }
    \]
     
    \end{enumerate}

   \section*{Выводы}
   
   В ходе выполнения лабораторной работы был закреплён перевод чисел из K-ичную СС, перевод в двоично-десятичную СС (8421, 8421+3 и 2421), способ сложения и умножения, а также изучен перевод в 3-ичную симметричную СС и Систему Остаточных Классов (СОК).
   
\end{document}