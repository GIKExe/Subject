\begin{enumerate}

\item[A)] Задана циклическая архитектура нейронной сети,
представляющая собой кольцо из N искусственных нейронов.
Каждый нейрон в определенный момент времени находится в
одном из двух состояний: активное (ненулевое значение уровня
активации) или молчащее (уровень активации равен 0). 
\\ Для поддержания стабильности работы сети реализован
механизм компенсации сигнала. Если какой-либо нейрон
находится в "молчащем" состоянии, его выходной сигнал
автоматически заменяется на сумму сигналов двух предыдущих
нейронов в кольце. Это позволяет сохранить информационный
поток и предотвратить "затухание" активности в сети.
Гарантируется, что размер архитектуры (количество нейронов
N) не менее трех. 

\item[B)] Во время обучения нейросети система мониторинга
записывает значения функции потерь (loss) после каждой эпохи.
Корректное обучение характеризуется строгим убыванием этих
значений — каждая следующая эпоха должна давать меньшую
ошибку, чем предыдущая. Однако в логах иногда появляются
отрицательные значения — это признак критического сбоя
(переполнение, сбой датчика или ошибки в pipeline). Всё,
что записано начиная с первого отрицательного числа и далее,
считается недостоверным и игнорируется. Дан массив из N целых
чисел — последовательность значений loss из лога. Определите,
образуют ли элементы до первого отрицательного числа строго
убывающую последовательность.

\newpage

\item[C)] В нейросети с бинарными нейронами (каждый нейрон
может находиться только в состоянии 0 — «выключен» или 1 —
«активен») исследователи анализируют последовательность
активаций одного нейрона во времени. Интерес представляют
непрерывные периоды активности — подряд идущие единицы. Чем
дольше нейрон остаётся активным без перерыва, тем выше
вероятность, что он участвует в важном когнитивном паттерне.
Вам дан временной ряд состояний нейрона длиной N : массив из
нулей и единиц. Требуется найти номер позиции (1-индексация),
с которой начинается самая длинная непрерывная
последовательность единиц. Если таких последовательностей
несколько, выберите последнюю (ту, что начинается позже всех).
Если в массиве вообще нет единиц, выведите два нуля.

\item[D)] Будем называть Проактивной нейронной сетью такую
сеть, в которой узлы постоянно оценивают "энергетический
потенциал" своих соседей. Необходимо реализовать алгоритм
первичной настройки такой сети. На вход подается
последовательность целых чисел, представляющая исходный
энергетический уровень каждого узла. Необходимо отфильтровать
и оставить только те узлы, которые обладают "стимулом к
росту", то есть те, чей энергетический потенциал меньше, чем
у узла справа. Такие узлы являются кандидатами для
формирования новых синаптических связей и будут
использоваться для построения следующего, более совершенного
слоя ИИ.

\item[E)] Система компьютерного зрения обрабатывает поток
кадров видео, где каждый кадр характеризуется объемом
вычислительной нагрузки (в условных единицах). Система
имеет ограниченную пропускную способность обработки — не
более K единиц нагрузки за один непрерывный временной
интервал. Определить максимальную продолжительность
непрерывного отрезка видео (количество подряд идущих кадров),
которую система может обработать без перегрузки, то есть
такую, чтобы суммарная нагрузка от этих кадров не превышала
лимит K.

\newpage

\item[F)] Перед вами – дамп активности искусственной нейронной
сети после прохождения тестового задания. Массив содержит
уровни возбуждения каждого из N нейронов. Требуется провести
диагностику:
\\ 1. Выявить "стабильные" нейроны. Нейрон считается
стабильным, если его уровень возбуждения кратен 3.
\\ 2. Оценить "фоновый шум". Усреднить уровень возбуждения
всех нейронов с чётными значениями. Найти целую часть
(округление по правилам арифметики) среднего арифметического
элементов с чётными значениями. Сформировать отчет число
стабильных нейронов необходимо внести в начало лога, а
усреднённый фон – в конец (увеличить массив на 2 элемента).

\item[H)] Вы проводите эксперименты по настройке
гиперпараметров модели машинного обучения. Матрица
представляет собой результаты N запусков с разными наборами
гиперпараметров. Каждая строка – значения функции потерь
(loss) на M последовательных эпохах обучения для одного
эксперимента. Среднее значение для каждой строки – это
устойчивость конфигурации гиперпараметров. Низкое среднее
значение loss означает, что данный набор параметров позволяет
модели эффективно обучаться. Определить наиболее
перспективная конфигурация гиперпараметров для дальнейшего
исследования и тонкой настройки (Наименьшее среди средних
значений).

\item[I)] ИИ-ассистент записывает лог операций с файлами,
где числа, отличные от нуля – размеры обработанных данных,
а нули – моменты отправки отчётов на сервер Между двумя
последними отчётами произошла утечка. Найти объём
потенциально утекших данных – сумму чисел между последними
нулями.

\newpage

\item[J)] При исследовании работы глубокой нейронной сети
производится замер уровней активации различных нейронов в
скрытых слоях. Особый интерес представляют паттерны активации,
которые могут свидетельствовать о специфических свойствах
обученной модели. Будем называть слой нейронов "резонансным",
если значения активаций его нейронов образуют пилообразный
паттерн, где каждый внутренний элемент либо больше, либо
меньше обоих соседних нейронов в слое. Такой паттерн может
указывать на: Наличие конкурентной активации между соседними
нейронами Специализацию нейронов на различных признаках
входных данных Эффект "расчесывания" признаков в процессе
обучения Определить является ли слой "резонансным".
Формализация: Дан массив из N вещественных чисел (N < 100),
где каждый элемент представляет уровень активации i-го
нейрона в исследуемом слое. Слой считается резонансным,
если для каждого внутреннего нейрона выполняется:
activations[i] > activations[i-1] и activations[i] >
activations[i+1] (доминирующая активация) ИЛИ activations[i]
< activations[i-1] и activations[i] < activations[i+1]
(подавленная активация)

\item[K)] ИИ-трейдер анализирует котировки акций. Массив
содержит цены закрытия за N периодов. Для трендового анализа
важно убедиться, что данные образуют неубывающую
последовательность (растущий тренд). Проверить, является
ли массив цен растущим трендом. Если да – вывести 0 (всё
корректно). Если нет – найти первую точку, где тренд сломался
(нумерация начинается с нуля).

\end{enumerate}