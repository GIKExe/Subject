\documentclass[oneside,a4paper,14pt]{extarticle} %размер шрифта 14
\usepackage[T1,T2A]{fontenc}
\usepackage[
	a4paper,
	letterpaper,
	top=2cm,
	bottom=2cm,
	left=2.5cm,
	right=1.5cm
]{geometry}
\usepackage[utf8]{inputenc} %кодировка текста
\usepackage[russian]{babel} %поддержка русского языка
\usepackage{textcomp} %текстовые символы
\usepackage{indentfirst} %корректировка отступов
\usepackage{listings} %для кода !!!
\usepackage{minted} %для кода !!!
\usepackage{graphicx} %работа с изображениям
\usepackage{mwe} % for blindtext and example-image-a in example
\usepackage{wrapfig}
\usepackage{caption}
\usepackage{amsmath}  % для формул и символов
\usepackage{amsfonts}
\usepackage{amsthm}
\usepackage{graphicx}
\usepackage[all]{xy}
\usepackage[breaklinks]{hyperref}
%%размеркегляузаголоковразделов 
\usepackage{titlesec} 
\titleformat{\section}
{\normalsize\bfseries} 
{\thesection} {1em}{} 
\titleformat{\subsection}
{\normalsize\bfseries}
{\thesubsection} {1em}{} 
\titleformat{\subsubsection}
{\normalsize\bfseries}
{\thesubsection} {1em}{}

\renewcommand\baselinestretch{1.33}\normalsize % межстрочный интервал
\setlength{\parindent}{1.25cm}
\usepackage{indentfirst}


\begin{document}
	\newpage\thispagestyle{empty}
	\begin{center}
		МИНИСТЕРСТВО НАУКИ И ВЫСШЕГО ОБРАЗОВАНИЯ\\
			РОССИЙСКОЙ ФЕДЕРАЦИИ
			ФЕДЕРАЛЬНОЕ ГОСУДАРСТВЕННОЕ БЮДЖЕТНОЕ\\
			ОБРАЗОВАТЕЛЬНОЕ
			УЧРЕЖДЕНИЕ ВЫСШЕГО ОБРАЗОВАНИЯ\\
			«ВЯТСКИЙ ГОСУДАРСТВЕННЫЙ УНИВЕРСИТЕТ»\\
			Институт математики и информационных систем\\
			Факультет автоматики и вычислительной техники\\
			Кафедра электронных вычислительных машин
	\end{center}
	\vspace{20mm}
	
	\begin{center}
		Отчёт по лабораторной работе №1\\
		по дисциплине\\
		<<Программирование>>\\
		<<Изучение условных и циклических структур в языке программирования>>\\
	\end{center}
	\vspace{48mm}
	
	Выполнил студент гр. ИВТб-1303-06-00 \hspace{10mm} \rule[-0,5mm]{23mm}{0.15mm}\,/Гортоломей И.К./
	
	
	Проверил преподаватель кафедры ЭВМ \hfill  \rule[-0,5mm]{30mm}{0.15mm}\,/Баташев П.А./
	
	\vfill
	\begin{center}
		Киров\\
		2025
	\end{center}

	\newpage%\thispagestyle{empty}
	
	\section{Цель} Цель лабораторной работы: закрепить на практике знания об основных конструкциях управления потоком выполнения программы, а именно о реализации разветвляющихся алгоритмов с помощью условных операторов и организации повторяющихся действий с помощью циклических структур.
	
	\section*{Задание}
	\begin{enumerate}
	
	\item Среди введенных N чисел определить длину максимальной возрастающей последовательности
	
	\item Для заданных натуральных чисел M и N. Получить сумму M младших цифр числа N.
	
	\item Будем называть трехзначное число "красивым", если полусумма его минимальной и максимальной цифры меньше оставшейся. Определите является ли введенное число "красивым".
	
	\item Среди произвольного количества целых чисел определить минимальный порядковый номер наименьшего из них.
	
	\item В некоторой стене осталось не закрытым прямоугольное отверстие размером А на В. Определить, проходит ли кирпич с размерами x, y, z через это отверстие.

	\item Заданы координаты вершин прямоугольника со сторонами, параллельными осям координат (x1,y1) и (x2,y2). Определить площадь части прямоугольника, расположенной в первой координатной четверти.

	\item Дана непустая последовательность ненулевых целых чисел. Определить, сколько раз в этой последовательности меняется знак. Например, в последовательности 1, -3, 8, 1, -5 знак меняется 3 раза.

	\item Необходимо протестировать группу из N человек. Каждый из них вводит: 1 – если он изучал английский язык, 2 – если немецкий, 3 – если французский, 0 – если не изучал никакой. Определите, сколько человек в каждой языковой группе.

	\item В катушке с автобусными билетами (номер билета шестизначный) меньший номер билета n, больший m. Определить количество счастливых билетов.

	\item В университете на потоке учатся M групп. Каждый месяц декан проводит конкурс на "хорошую" группу. Для этого оценивается число пропущенных занятий каждым студентом группы. и рассчитывается среднее значение по группе Nm, где m номер группы. Если минимальное число пропусков N1, N2, N3, N4...Nm меньше 10, то на потоке «Есть хорошая группа». Помогите декану провести конкурс. Если хорошая группа найдется выведите сообщение «The good group» и укажите ее номер. Если такой группы нет выведете "No"

	\item Заданы $k_1,\ b_1,\ k_2,\ b_2$ и e (e > 0). Определить, находится ли точка пересечения прямых заданных уравнениями $y=k_1\cdot x+b_1$ и $y= k_2\cdot x+b_2$ на расстоянии не более e от начала координат.

	\item Дано натуральное число n. Проверить, является ли оно совершенным (число называется совершенным, если оно равно сумме всех своих делителей)

	\end{enumerate}

	\newpage
	\section*{Решение}
	\begin{enumerate}
		\item задание (на Си) \\
		\begin{minipage}[t]{0.68\textwidth}
			 \includegraphics[width=\linewidth]{задание 1.png}
		\end{minipage}
		
		\newpage
		\begin{minted}{C}
#include <stdio.h>

int main() {
  int N, X, OX;
  int MM = 1;
  int M = 1;

  scanf("%d", &N);
  scanf("%d", &OX);
  N--;

  for (; N > 0; N--) {
	scanf("%d", &X);
	if (X > OX) {
	  M++;
	  if (M > MM) MM = M;
	} else {
	  M = 1;
	}
	OX = X;
  }
  printf("%d", MM);
  return 0;
}
		\end{minted}
		\newpage

		\item задание (на Паскале) \\
		\includegraphics[width=16cm]{задание 2.png}
		\begin{minted}{Pascal}
program Main;
  var M, N, X, i: LongInt;
begin
  X := 0;
  readln(M);
  readln(N);
  for i := 1 to N do begin
	X := X + M mod 10;
	M := M div 10;
  end;
  writeln(X);
end.
		\end{minted}
		\newpage

		\item задание (на Си) \\
		\includegraphics[width=16cm]{задание 3.png}
		\newpage
		\begin{minted}{C}
#include <stdio.h>

int main() {

  short x;
  scanf("%hi", &x);

  char a, b, c, s;
  a = x % 10;
  b = x / 10 % 10;
  c = x / 100;

  if (a > b) { s = a; a = b; b = s; }
  if (b > c) { s = b; b = c; c = s; }
  if (a > b) { s = a; a = b; b = s; }

  if ((a+c)/2.0 < b) {
	printf("YES");
  } else {
	printf("NO");
  }
  return 0;
}
		\end{minted}
		\newpage

		\item задание (на Паскале) \\
		 \includegraphics[width=16cm]{задание 4.png}
		\newpage
		
		\begin{minted}{Pascal}
program Main;
  var x, min_index, min, index: Integer;
begin
  index := 0;
  min_index := 0;
  while true do begin
	read(x);
	if x = 0 then break;
	if index = 0 then min := x + 1;
	if x < min then begin
	  min := x;
	  min_index := index;
	end;
	index := index + 1;
  end;
  writeln(min_index);
end.
		\end{minted}
		\newpage

		\item задание (на Си) \\
		 \includegraphics[width=12cm]{задание 5.png}

		\begin{minted}{C}
#include <stdio.h>  

int main() {  
  unsigned int a, b, x, y, z;  
  scanf("%d %d %d %d %d", &a, &b, &x, &y, &z);  
  if (
	(x*y <= a*b) || (x*z <= a*b) || (z*y <= a*b)
  ) {
	printf("Yes");
  } else {
	printf("No");
  }
  return 0;  
} 
		\end{minted}
		\newpage
		
		\item задание (на Паскале) \\
		\includegraphics[width=15cm]{задание 6.png}
		\begin{minted}{Pascal}
program Main;
  var x1, y1, x2, y2: LongInt;
begin
  read(x1, y1, x2, y2);
  if x1 < 0 then x1 := 0;
  if x2 < 0 then x2 := 0;
  if y1 < 0 then y1 := 0;
  if y2 < 0 then y2 := 0;
  write(abs(x1 - x2) * abs(y1 - y2));
end.
		\end{minted}
		\newpage

		\item задание (на Си) \\
		\includegraphics[width=15cm]{задание 7.png}
		\newpage
		\begin{minted}{C}
#include <stdio.h>

int main() {
  long n, res, last, num;
  scanf("%ld", &n);
  scanf("%ld", &last);
  res = 0;
  for (; n > 1; n--) {
	scanf("%ld", &num);
	if (num*last < 0) res++;
	last = num;
  }
  printf("%ld", res);
}
		\end{minted}
		\newpage

		\item задание (на Паскале) \\
		\includegraphics[width=16cm]{задание 8.png}
		\newpage
		\begin{minted}{Pascal}
program Main;
  var x: Byte;
  var i, n, a, b, c: Word;
begin
  readln(n);
  for i := 1 to n do begin
	readln(x);
	if x = 1 then a := a+1;
	if x = 2 then b := b+1;
	if x = 3 then c := c+1;
  end;
  writeln(a);
  writeln(b);
  writeln(c);
end.
		\end{minted}
		\newpage

		\item задание (на Паскале) \\
		\includegraphics[width=16cm]{задание 9.png}
		\newpage
		\begin{minted}{Pascal}
program Main;
  var x, y, res: Word;
  var a, b: LongWord;
begin
  res := 0;
  read(a, b);
  while a <= b do begin
	x := a mod 1000;
	y := a div 1000;
	x := x mod 10 + (x div 10) mod 10 + x div 100;
	y := y mod 10 + (y div 10) mod 10 + y div 100;
	if x = y then res := res + 1;
	a := a + 1;
  end;
  writeln(res);
end.
		\end{minted}
		\newpage

		\item задание (на Си) \\
		\includegraphics[width=16cm]{задание 10.png}
		\newpage
		\begin{minted}{C}
#include <stdio.h>

int main() {
  unsigned short m, n, m2, n2, sum, x;
  scanf("%hu", &m);
	m2 = m+1;
  for (; m > 0; m--) {
	scanf("%hu", &n);
	sum = 0;
	n2 = n;
	for (; n > 0; n--) {
	  scanf("%hu", &x);
	  sum += x;
	}
	if (sum / (float) n2 < 10) {
	  printf("The good group %hu", m2 - m);
	  return 0;
	}
  }
  printf("No");
  return 0;
}
		\end{minted}
		\newpage

		\item задание (на Паскале) \\
		\includegraphics[width=14cm]{задание 11.png}
		\newpage
		\begin{minted}{Pascal}
program Main;
  var k1, b1, k2, b2: LongInt;
  var x, y, e: Double;
begin
  read(k1, b1, k2, b2, e);
  if (k1 = k2) then begin
	if (b1 = b2) then
	  write('Yes')
	else
	  write('No');
  end else begin
	x := (b2 - b1) / (k1 - k2);
	y := k1 * x + b1;
	if sqrt(sqr(x) + sqr(y)) < e then
	  write('Yes')
	else
	  write('No');
  end;
end.
		\end{minted}
		\newpage

		\item задание (на Си) \\
		\includegraphics[width=14cm]{задание 12.png}
		\begin{minted}{C}
#include <stdio.h>

int main() {
  unsigned short n, i;
  unsigned int sum = 0; 
  scanf("%hu", &n);
  for (i = 1; i < n; i++) {
	if (n%i == 0) sum += i;
  }
  if (n == sum) {printf("YES");} else {printf("NO");}
  return 0;
}
		\end{minted}
	\end{enumerate}
	
	\section*{Выводы}
	В ходе лабораторной работы были успешно закреплены на практике знания об основных конструкциях управления потоком выполнения программы. При решении задач были применены условные операторы для реализации разветвляющихся алгоритмов и циклические структуры для организации повторяющихся действий, что позволило отработать принципы построения алгоритмов с использованием этих конструкций.
\end{document}